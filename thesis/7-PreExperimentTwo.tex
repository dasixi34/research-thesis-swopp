\section{事前実験2}
\label{sec:PreExperimentTwo}
本章では,提案手法の有効性を確認するために行った事前実験2について述べる.事前実験2では,送られてきた32個の値を一定回数足し合わせて送り返すプログラムにおいて,TCP/IPによる通信を用いる場合と提案手法を用いる場合の通信のスループットを比較した.

\subsection{事前実験2用のプログラム}
事前実験2用のプログラムとして,クライアントから送られてきた32個の値を一定回数足し合わせて送り返すプログラムを作成した(図\ref{fig:PreExperimentTwo}).DPDKのRun-to-Completionモデルで実行する処理は受信したパケットに含まれる32個の値を足し合わせる処理である.このプログラムはTCP/IPによる通信を用いるものと提案手法を用いるものの2種類を作成した.

\begin{figure}[htb]
  \centering
  \includegraphics[width=\columnwidth]{pictures/PreExperimentTwo.pdf}
  \caption{事前実験2用のプログラム}
  \label{fig:PreExperimentTwo}
\end{figure}

\subsection{実験環境}
事前実験2で用いたネットワーク構成は事前実験1で用いたもの(図\ref{fig:PreExperimentNetwork})と同じである.ただし,クライアントにはパケットサイズが64バイトのパケットを送信レート100\%で送信し続ける自作プログラムを用いた.送信されるパケットのペイロードは要素数が32でデータ型がuint16\_tの配列である.なお,事前評価2で用いた計算機の性能は事前実験1で用いたもの(表\ref{tab:MachineSpec})と同じである.

\subsection{実験結果・考察}
事前実験2の結果を図\ref{fig:PreEvaluationTwoResult}に示す.このグラフの横軸は受信した32個の値を足し合わせる回数,縦軸は通信のスループットを表している.また,青いバーはTCP/IPによる通信を用いた場合,赤いバーは提案手法を用いた場合を表している.グラフより,提案手法を用いた場合のスループットはTCP/IPによる通信を用いた場合に比べて最大30倍程度高いことがわかる.よって,分散計算環境において,DPDKによるL2通信を用いることとDPDKのRun-to-Completionモデルで処理を行うことは有効であると考える.なお,提案手法を用いた場合のスループットが足し合わせる回数が多くなるにつれて下がっていくのは,計算処理が重たくなり受信処理にCPUリソースが割当たらなくなったためである.

\begin{figure}[htb]
  \centering
  \includegraphics[width=\columnwidth]{pictures/PreExperimentTwoResult.pdf}
  \caption{事前実験2の結果}
  \label{fig:PreEvaluationTwoResult}
\end{figure}
