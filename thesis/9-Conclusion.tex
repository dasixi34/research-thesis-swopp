\section{まとめと今後の課題}
\label{sec:Conclusion}
本稿では,DPDKのRun-to-Completionモデルを用いたL2分散計算環境を提案した.計算機間の通信にDPDKによるL2通信を用いることによって,TCP/IPによる通信のオーバーヘッドを低減するとともに,カーネルによるパケットI/O処理に比べて高速なパケットI/O処理を用いることができる.また,Run-to-Completionモデルを採用することによって,Pipelineモデルを用いる場合に比べてCPUリソースやL1キャッシュを有効活用できる.送られてきた32個の値を加算し送り返すプログラムを用いた事前実験2では,提案手法を用いた場合の処理性能はTCP/IPによる通信を用いた場合に比べて,内部ループ回数が1回の場合は30倍,10回の場合は4倍,100回の場合は0.4倍高いことを確認した.また,単回帰分析を行うプログラムを用いた評価では,提案手法を用いた2台での分散学習の実行時間は,1台での集中学習の場合に比べて2倍,TCP/IPによる通信を用いた2台での分散学習の場合に比べて1.2倍高速であることを確認した.

今後の課題としては,使用する計算機の台数を増やして評価を取ること,提案手法のどの部分が性能向上に寄与しているのかを調べるためにRaw Socketを用いてカーネルによるL2通信を実装して評価を取ること,ロジスティック回帰やサポートベクターマシン(Support Vector Machine, SVM)といった複雑な機械学習を実行することなどがある.
