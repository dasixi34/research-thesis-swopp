\section{はじめに}
\label{sec:Background}
分散計算環境で実行する処理の中には,与えられたデータを繰り返し用いながら,小さいデータを計算機間でやりとりする処理形態が存在する.例えば,単回帰分析やロジスティック回帰などの機械学習である.これらの処理において,TCP/IPによる制御は計算機内での本来の演算処理時間に比べて,かなり大きなオーバーヘッドとなる.また,カーネルによるパケットI/O処理は,一定時間に受信するパケットの量が増えると,コンテキストスイッチが増加して,割り込み以外の処理時間が相対的に少なくなるため,DPDKのパケットI/O処理に比べて低速となる.

DPDKにはRun-to-CompletionモデルとPipelineモデルの2つのモデルがある.Run-to-Completionモデルは受信処理,パケット処理,送信処理を一つの論理コアで行い,Pipelineモデルは受信処理,パケット処理,送信処理をそれぞれ別の論理コアで行う.Pipelineモデルはそれぞれの処理が論理コアを専有するため,CPUリソースやL1キャッシュを有効活用できない.

そこで本研究では,DPDKのRun-to-Completionモデルを用いたL2分散計算環境を提案する.計算機間の通信にDPDKによるL2通信を用いることによって,TCP/IPによる通信のオーバーヘッドを低減するとともに,カーネルによるパケットI/O処理に比べて高速なパケットI/O処理を用いることができる.また,Run-to-Completionモデルを採用することによって,Pipelineモデルを用いる場合に比べてCPUリソースやL1キャッシュを有効活用できる.

なお,本研究が提案する分散計算環境には3つの前提を設ける.1つ目の前提はL2通信が可能であるローカルなクラスタ環境で動作することである.2つ目の前提は計算機間でやりとりされるデータのサイズはL2フレームより小さいことである.3つ目の前提は各計算機で実行される処理はイテレーションが多用され,かつ,その実行に必要なデータは小規模であることである.これらの前提は特に機械学習では満たされることが多いと考える.

本稿の構成を以下に示す.第\ref{sec:Problem}章で分散計算環境の問題点を述べ,第\ref{sec:DPDK}章でDPDKについて説明する.第\ref{sec:Proposed}章で提案手法について述べ,第\ref{sec:PreExperimentOne}章,第\ref{sec:PreExperimentTwo}章,第\ref{sec:Evaluation}章で提案手法の事前実験と評価を行う.第\ref{sec:RelatedWorks}章で関連研究を述べ,第\ref{sec:Conclusion}章でまとめと今後の課題を述べる.
