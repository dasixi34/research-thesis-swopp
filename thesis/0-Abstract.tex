\begin{abstract}
  分散計算環境で実行される処理の中には,与えられたデータを繰り返し用いながら,L2フレームより小さいデータを計算機間でやりとりする処理がある.例えば,単回帰分析やロジスティック回帰などの機械学習である.これらの処理において,TCP/IPによる制御はオーバーヘッドである.また,カーネルによるパケットI/O処理はDPDKのパケットI/O処理に比べて低速である.DPDKにはRun-to-CompletionモデルとPipelineモデルの2つのモデルがある.しかし,PipelineモデルはCPUリソースやL1キャッシュを有効活用できない.本研究では,DPDKのRun-to-Completionモデルを用いたL2分散計算環境を提案する.計算機間の通信にDPDKによるL2通信を用いることによって,TCP/IPによる通信のオーバーヘッドを低減するとともに,カーネルによるパケットI/O処理に比べて高速なパケットI/O処理を用いることができる.また,Run-to-Completionモデルを採用することによって,Pipelineモデルを用いる場合に比べてCPUリソースやL1キャッシュを有効活用できる.送られてきた32個の値を一定回数足し合わせて送り返すプログラムを用いた事前実験2では,提案手法を用いた場合の処理性能はTCP/IPによる通信を用いた場合に比べて,足し合わせる回数が1回の場合は30倍,10回の場合は4倍,100回の場合は0.4倍高いことを確認した.また,単回帰分析を行うプログラムを用いた評価では,提案手法を用いた2台での分散学習の実行時間は,1台での集中学習の場合に比べて2倍,TCP/IPによる通信を用いた2台での分散学習の場合に比べて1.2倍高速であることを確認した.
\end{abstract}

\begin{jkeyword}
  分散計算環境,DPDK,Run-to-Completion,L2通信,L1キャッシュ
\end{jkeyword}
