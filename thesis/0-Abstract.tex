\begin{abstract}
  分散計算環境で実行する処理の中には,与えられたデータを繰り返し用いながら,小さいデータを計算機間でやりとりする処理がある.例えば,単回帰分析やロジスティック回帰などの機械学習である.これらの処理において,TCP/IPによる制御は相対的に大きなオーバーヘッドとなる.また,カーネルによるパケットI/O処理はDPDKのパケットI/O処理に比べて低速である.本研究では,DPDKのRun-to-Completionモデルを用いたL2分散計算環境を提案する.計算機間の通信にDPDKによるL2通信を用いることによって,TCP/IPによる通信のオーバーヘッドを低減するとともに,カーネルによるパケットI/O処理に比べて高速なパケットI/O処理を用いることができる.また,Run-to-Completionモデルを採用することによって,Pipelineモデルを用いる場合に比べてCPUリソースやL1キャッシュを有効活用できる.送られてきた32個の値を加算し送り返すプログラムを用いた事前実験2では,提案手法を用いた場合の処理性能はTCP/IPによる通信を用いた場合に比べて,内部ループ回数が1回の場合は30倍,10回の場合は4倍,100回の場合は0.4倍高いことを確認した.また,単回帰分析を行うプログラムを用いた評価では,提案手法を用いた2台での分散処理の実行時間は,1台での集中学習の場合に比べて2倍,TCP/IPによる通信を用いた2台での分散処理の場合に比べて1.2倍高速であることを確認した.
\end{abstract}

\begin{jkeyword}
  分散計算環境,DPDK,Run-to-Completion,L2通信,L1キャッシュ
\end{jkeyword}
