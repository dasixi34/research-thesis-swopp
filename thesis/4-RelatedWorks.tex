\section{関連研究}
\label{sec:RelatedWorks}
多元連立一次方程式の緩和法解析の分散処理にDPDKによる通信を用いる研究 \cite{RelaxationMethodDPDK} がある.1992年に上智大学の小石らによって行われた多元連立一次方程式の緩和法解析の分散処理に関する研究 \cite{RelaxationMethodUDP} では,UDPによるブロードキャストが行われていた.そこで,この研究では多元連立一次方程式の緩和法解析の分散処理にDPDKを用いることで,通信オーバーヘッドの削減による計算の高速化を行った.その結果,DPDKを用いた緩和法解析の分散処理アプリケーションは,UDPを用いた分散処理よりも最大40.1\%高速化された.しかし,この研究では受信スレッド,収束判定または計算のスレッド,送信スレッドのそれぞれが論理コアを使用するPipelineモデルを用いているため,コアの使用効率が悪い問題とCPUのL1キャッシュを有効活用できない問題がある.

MPI通信のデータ転送にDPDKを用いる研究 \cite{MPIDPDK} もある.多くのMPIライブラリはデータ転送を行うレイヤが独立しており,ユーザの環境に合わせてデータ転送方式やデバイスをMPIプログラムの実行に柔軟に切り替えることができる.そこで,この研究はMPIライブラリの新たなデータ転送モジュールとしてDPDKによるデータ転送モジュールを提案した.通信のスループットが低いときはRun-to-Completionモデル,高いときはPipelineモデルを使用するようになっている.その結果,TCP/IPソケットによるデータ転送を用いた場合と比べ,通信遅延を最大77\%改善することができた.しかし,この研究ではACKパケットの授受を実装することによって,パケットロスに対する制御を行っているため,通信のオーバーヘッドがある.
